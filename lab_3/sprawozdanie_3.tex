%%%%%%%%%%%%%%%%%%%%%%%%%%%%%%%%%%%%%%%%%%%%%%%%%%%%%%%%%%%%%%%%%%%%%%%%%%%%%%%
%
%   UNIWERSALNY SZABLON SPRAWOZDANIA - OBLICZENIA NAUKOWE
%   Autor: Marcel Musiałek (279704)
%
%%%%%%%%%%%%%%%%%%%%%%%%%%%%%%%%%%%%%%%%%%%%%%%%%%%%%%%%%%%%%%%%%%%%%%%%%%%%%%%

% --- 1. USTAWIENIA DOKUMENTU ---
\documentclass[11pt, a4paper]{article}

% --- 2. PAKIETY JĘZYKOWE I KODOWANIE ---
\usepackage[utf8]{inputenc}     % Kodowanie pliku (UTF-8)
\usepackage[T1]{fontenc}        % Kodowanie czcionek (dla polskich znaków w PDF)
\usepackage{lmodern}            % Ładniejsza, wektorowa czcionka
\usepackage[polish]{babel}      % Polskie nazwy (np. "Spis treści", "Rysunek")

% --- 3. GEOMETRIA STRONY ---
\usepackage{geometry}
\geometry{
    a4paper,
    margin=2.5cm,               % Marginesy 2.5cm z każdej strony
    top=2.5cm,
    bottom=2.5cm
}

% --- 4. MATEMATYKA I KOD ---
\usepackage{amsmath}            % Rozszerzone wzory matematyczne
\usepackage{amssymb}            % Symbole matematyczne
\usepackage{verbatim}           % Środowisko do wklejania outputu z konsoli
\usepackage{graphicx}           % Obsługa obrazków (gdybyś chciał wstawić wykres)
\usepackage{float}              % Lepsze pozycjonowanie obrazków [H]

% --- 5. LINKI I ZAKŁADKI (Hyperref) ---
% Ładujemy na końcu, aby uniknąć konfliktów.
% Opcja unicode=true pomaga z polskimi znakami w zakładkach.
\usepackage[unicode=true, colorlinks=true, linkcolor=black, urlcolor=blue, pdfusetitle]{hyperref}

% --- 6. WŁASNE KOMENDY ---
% Linia oddzielająca zadania (bezpieczna wersja)
\newcommand{\taskseparator}{\vspace{0.5cm}\leavevmode\leaders\hrule height 0.5pt\hfill\kern0pt\vspace{0.5cm}}

% Komenda do formatowania nazw zmiennych w tekście (np. \code{macheps})
\newcommand{\code}[1]{\texttt{#1}}


\title{\textbf{Sprawozdanie z Laboratorium obliczeń naukowych} \\ Lista nr [3]} 
\author{Marcel Musiałek \\ Indeks: 279704}
\date{\today}


%%%%%%%%%%%%%%%%%%%%%%%%%%%%%%%%%%%%%%%%%%%%%%%%%%%%%%%%%%%%%%%%%%%%%%%%%%%%%%%
%
%   TREŚĆ DOKUMENTU
%
%%%%%%%%%%%%%%%%%%%%%%%%%%%%%%%%%%%%%%%%%%%%%%%%%%%%%%%%%%%%%%%%%%%%%%%%%%%%%%%

\begin{document}

% Generowanie strony tytułowej
\maketitle

% Generowanie spisu treści (wymaga dwukrotnej kompilacji)
\tableofcontents
\newpage

% --- ZADANIA 1-3 ---
\section{Zadania 1--3: Implementacja i weryfikacja metod}

\subsection{Opis zadań}
Celem zadań było zaimplementowanie metod: bisekcji, Newtona oraz siecznych, zgodnie z pseudokodem przedstawionym na wykładzie. Następnie należało sprawdzić poprawność implementacji przy użyciu prostych funkcji testowych.

\subsection{Opis rozwiązania}
W ramach rozwiązania stworzono funkcje realizujące poszczególne metody. W celu weryfikacji ich działania dodano proste testy dla funkcji $f(x) = x^2 - 4$.

\subsection{Wyniki}
\begin{verbatim}
--- WERYFIKACJA NA FUNKCJI f(x) = x^2 - 4 ---
Bisekcja [1.0, 3.0]:     (2.0, 0.0, 1, 0)
Newton (x0=3.0):         (2.000000000026214, 1.0485656787295738e-10, 4, 0)
Sieczne (x0=1.0, x1=3.0):(2.000000195092221, 7.803689223706556e-7, 5, 0)
\end{verbatim}

\subsection{Wnioski}
Wyniki testu potwierdzają, że zaimplementowane funkcje działają poprawnie dla prostych danych wejściowych, zbiegając do oczekiwanego rozwiązania.

\taskseparator


% --- ZADANIE 4 ---
\section{Zadanie 4}

\subsection{Opis zadania}
Celem zadania było wyznaczenie pierwiastka równania $f(x)$ przy użyciu trzech zaimplementowanych metod iteracyjnych:
\begin{itemize}
    \item Metody bisekcji,
    \item Metody Newtona,
    \item Metody siecznych.
\end{itemize}

\subsection{Opis rozwiązania}
Wykorzystano moduł miejsca zerowe zawierający implementacje metod. Zdefiniowano funkcję $f(x)$ oraz jej pochodną $f'(x)$. Dla każdej metody wywołano odpowiednią funkcję z zadanymi parametrami i zmierzono liczbę iteracji oraz końcowy błąd $f(r)$.

\subsection{Wyniki}
\begin{verbatim}
    Wymagana dokładność: delta = 5.0e-6, epsilon = 5.0e-6

    1. Metoda Bisekcji (przedział [1.5, 2.0]):
    Pierwiastek r = 1.933753967285156
    Wartość f(r)  = -2.702768013840284e-07
    Iteracje      = 16
    ------------------------------------------------------------
    2. Metoda Newtona (x0 = 1.5):
    Pierwiastek r = 1.933753779789742
    Wartość f(r)  = -2.242331631485683e-08
    Iteracje      = 4
    ------------------------------------------------------------
    3. Metoda Siecznych (x0 = 1.0, x1 = 2.0):
    Pierwiastek r = 1.933753644474301
    Wartość f(r)  = 1.564525129449379e-07
    Iteracje      = 4
    ============================================================
\end{verbatim}

\subsection{Obserwacje}
\begin{itemize}
    \item Bardzo podobne miejsca zerowe.
    \item Większa ilość iteracji w metodzie bisekcji.
    \item Różne znaki $f(r)$.
\end{itemize}

\subsection{Wnioski}
Ten eksperyment potwierdza teoretyczne właściwości badanych metod:
\begin{itemize}
    \item \textbf{Metoda Bisekcji} – Jest zdecydowanie najwolniejsza, posiada zbieżność liniową.
    \item \textbf{Metoda Newtona} – Ma teoretycznie najlepszą zbieżność (kwadratową) – w eksperymencie również daje najlepsze wyniki.
    \item \textbf{Metoda Siecznych} – Uzyskała równie dobre wyniki co metoda Newtona, mimo teoretycznie gorszej zbieżności.
\end{itemize}
Eksperyment ten ukazuje, że na określonym wąskim przedziale lepiej sprawdzają się metody Newtona i siecznych. Pokazuje to, dlaczego metody hybrydowe są skuteczne.

\taskseparator

% --- ZADANIE 5 ---
\section{Zadanie 5}

\subsection{Opis zadania}
Zadanie polegało na znalezieniu wartości $x$, dla których przecinają się wykresy funkcji $y = 3x$ oraz $y = e^x$.

\subsection{Opis rozwiązania}
Sprawdzenie szerokiego zakresu dla metody bisekcji $[-100, 100]$ – szansa, że algorytm i tak znajdzie rozwiązanie.
Zauważyć można, że $f(0) = -1$, $f(1) < 0$, $f(2) > 0$, co oznacza, że w przedziałach $[0, 1]$ i $[1, 2]$ znajdują się miejsca zerowe.
Zaimplementowano skrypt wykorzystujący funkcję mbisekcji z modułu \code{MiejscaZerowe}, uruchamiając ją niezależnie dla obu wyznaczonych wyżej przedziałów.

\subsection{Wyniki}
\begin{verbatim}
Szukanie w przedziale [-100.0, 100.0]:
   Błąd: Funkcja nie zmienia znaku w zadanym przedziale.

Szukanie w przedziale [0.0, 1.0]:
   Pierwiastek r = 0.61914062
   Wartość f(r)  = 9.06632034e-05
   Liczba iteracji: 9

Szukanie w przedziale [1.0, 2.0]:
   Pierwiastek r = 1.51208496
   Wartość f(r)  = 7.61857860e-05
   Liczba iteracji: 13
\end{verbatim}

\subsection{Obserwacje}
\begin{itemize}
    \item W szerokim zakresie nie działa metoda – ustalono, że funkcja ma 2 pierwiastki, więc na końcach szerokiego zakresu znak będzie taki sam.
    \item W obu przybliżonych przypadkach pierwiastek funkcji został znaleziony z rozsądną ilością iteracji.
\end{itemize}

\subsection{Wnioski}
\begin{itemize}
    \item Metoda bisekcji wymaga zmiany znaku na danym zakresie, więc w przypadku wystąpienia na nim parzystej ilości pierwiastków metoda może zawieść.
    \item Na przedziałach, gdzie zmiana znaku występuje raz, metoda bisekcji gwarantuje znalezienie pierwiastka.
\end{itemize}

\taskseparator


% --- ZADANIE 6 ---
\section{Zadanie 6}

\subsection{Opis zadania}
Zadanie polegało na znalezieniu miejsc zerowych dwóch zadanych funkcji.
Celem było przetestowanie metod Bisekcji, Newtona i Siecznych dla standardowych przedziałów, a następnie zbadanie zachowania Metody Newtona dla "trudnych" punktów startowych, gdzie pochodna jest bliska zeru lub zmienia znak.

\subsection{Opis rozwiązania}
Zaimplementowano funkcje $f_1, f_2$ oraz ich pochodne.
\begin{itemize}
    \item Dla $f_1$ przeprowadzono testy stabilności Newtona dla dużych $x_0$, gdzie wykres funkcji staje się płaski (asymptota pozioma).
    \item Dla $f_2$ sprawdzono punkt $x_0 = 1$ (ekstremum lokalne) oraz punkty $x_0 > 1$, gdzie styczna kieruje iteracje w przeciwną stronę od pierwiastka.
\end{itemize}

\subsection{Wyniki}
\begin{verbatim}
Analiza funkcji f1(x) = e^(1-x) - 1
------------------------------------------------------------
Bisekcja [0.0, 2.0]:     r=1.000000, it=1, err=0
Newton (x0=2.0):         r=1.000000, it=5, err=0
Sieczne (x0=0.0, x1=2.0): r=1.000002, it=6, err=0

Testy specjalne Newtona dla f1 (x0 > 1):
x0 = 5.0   -> r=1.000000, it=54, v=3.6e-07
x0 = 10.0  -> r=NaN, it=100, v=NaN
x0 = 100.0 -> r=100.000000, it=1, v=-1.0e+00

Analiza funkcji f2(x) = x * e^(-x)
------------------------------------------------------------
Bisekcja [-0.5, 0.5]:    r=0.000000, it=1, err=0
Newton (x0=-0.5):        r=-0.000000, it=4, err=0
Sieczne (x0=-0.5, x1=0.5): r=0.000005, it=6, err=0

Testy specjalne Newtona dla f2 (szukanie zera w x=0):
x0 = 1.0 (ekstremum!): err kod = 2 (oczekiwany błąd pochodnej)
x0 = 1.5 -> r=14.787437, it=10 (czy zbiegła do 0?)
x0 = 2.0 -> r=14.398663, it=10 (czy zbiegła do 0?)
x0 = 10.0 -> r=14.380524, it=4 (czy zbiegła do 0?)
\end{verbatim}

\subsection{Obserwacje}
\textbf{Dla funkcji $f_1$:}
\begin{itemize}
    \item Metody zbiegły szybko. Bisekcja trafiła idealnie w 1 iteracji, ponieważ $x=1$ jest środkiem przedziału $[0, 2]$.
    \item Dla $x=5$ pochodna jest bardzo mała (funkcja płaska). Metoda Newtona potrzebowała aż 54 iteracji, aby "wrócić" do pierwiastka.
    \item Dla bardzo dużych $x$, pochodna jest numerycznie zerowa. Metoda albo generuje błędy (\code{NaN}), albo stoi w miejscu, ponieważ styczna jest prawie równoległa do osi OX.
\end{itemize}

\noindent \textbf{Dla funkcji $f_2$:}
\begin{itemize}
    \item Funkcja ma maksimum w $x=1$. Pochodna wynosi 0. Metoda Newtona zwróciła błąd (dzielenie przez zero), co jest zachowaniem poprawnym.
    \item Dla $x > 1$ styczna do wykresu przecina oś OX po prawej stronie (w coraz większych wartościach $x$), oddalając się od pierwiastka $x=0$. Metoda rozbiega się do nieskończoności. Wartości typu $14.7$ to moment, w którym algorytm się zatrzymał (przypadkowo spełniając warunek tolerancji dla bardzo małej wartości funkcji), ale nie jest to poszukiwany pierwiastek.
\end{itemize}

\taskseparator

\end{document}