%%%%%%%%%%%%%%%%%%%%%%%%%%%%%%%%%%%%%%%%%%%%%%%%%%%%%%%%%%%%%%%%%%%%%%%%%%%%%%%
%
%   UNIWERSALNY SZABLON SPRAWOZDANIA - Obliczenia Naukowe
%   Autor: Marcel Musiałek (279704)
%
%%%%%%%%%%%%%%%%%%%%%%%%%%%%%%%%%%%%%%%%%%%%%%%%%%%%%%%%%%%%%%%%%%%%%%%%%%%%%%%

% --- 1. USTAWIENIA DOKUMENTU ---
\documentclass[11pt, a4paper]{article}

% --- 2. PAKIETY JĘZYKOWE I KODOWANIE ---
\usepackage[utf8]{inputenc}     % Kodowanie pliku (UTF-8)
\usepackage[T1]{fontenc}        % Kodowanie czcionek (dla polskich znaków w PDF)
\usepackage{lmodern}            % Ładniejsza, wektorowa czcionka
\usepackage[polish]{babel}      % Polskie nazwy (np. "Spis treści", "Rysunek")

% --- 3. GEOMETRIA STRONY ---
\usepackage{geometry}
\geometry{
    a4paper,
    margin=2.5cm,               % Marginesy 2.5cm z każdej strony
    top=2.5cm,
    bottom=2.5cm
}

% --- 4. MATEMATYKA I KOD ---
\usepackage{amsmath}            % Rozszerzone wzory matematyczne
\usepackage{amssymb}            % Symbole matematyczne
\usepackage{verbatim}           % Środowisko do wklejania outputu z konsoli
\usepackage{graphicx}           % Obsługa obrazków
\usepackage{float}              % Lepsze pozycjonowanie obrazków [H]
\usepackage{enumitem}           % Lepsze formatowanie list
\usepackage{subcaption}         % Umożliwia grupowanie obrazków (subfigures)

% --- 5. LINKI I ZAKŁADKI (Hyperref) ---
\usepackage[unicode=true, colorlinks=true, linkcolor=black, urlcolor=blue, pdfusetitle]{hyperref}

% --- 6. WŁASNE KOMENDY ---
\newcommand{\taskseparator}{\vspace{0.5cm}\leavevmode\leaders\hrule height 0.5pt\hfill\kern0pt\vspace{0.5cm}}
\newcommand{\code}[1]{\texttt{#1}}


%%%%%%%%%%%%%%%%%%%%%%%%%%%%%%%%%%%%%%%%%%%%%%%%%%%%%%%%%%%%%%%%%%%%%%%%%%%%%%%
%
%   DANE AUTORA
%
%%%%%%%%%%%%%%%%%%%%%%%%%%%%%%%%%%%%%%%%%%%%%%%%%%%%%%%%%%%%%%%%%%%%%%%%%%%%%%%

\title{\textbf{Sprawozdanie z Laboratorium Obliczeń Naukowych} \\ Lista nr 4: Interpolacja}
\author{Marcel Musiałek \\ Indeks: 279704}
\date{\today}


%%%%%%%%%%%%%%%%%%%%%%%%%%%%%%%%%%%%%%%%%%%%%%%%%%%%%%%%%%%%%%%%%%%%%%%%%%%%%%%
%
%   TREŚĆ DOKUMENTU
%
%%%%%%%%%%%%%%%%%%%%%%%%%%%%%%%%%%%%%%%%%%%%%%%%%%%%%%%%%%%%%%%%%%%%%%%%%%%%%%%

\begin{document}

\maketitle
\tableofcontents
\newpage

% -----------------------------------------------------------------------------
% ZADANIA 1-4
% -----------------------------------------------------------------------------

\section{Implementacja Metod (Zadania 1--4)}

Stworzono moduł \code{Interpolacja} w języku Julia, zawierający następujące funkcje:

\subsection{Zadanie 1: Ilorazy różnicowe}
\begin{itemize}
    \item \textbf{Funkcja:} \code{ilorazyRoznicowe(x, f)}
    \item \textbf{Opis:} Oblicza współczynniki wielomianu Newtona.
    \item \textbf{Optymalizacja:} Zgodnie z poleceniem, nie użyto macierzy dwuwymiarowej. Algorytm działa na wektorze jednowymiarowym, aktualizując go ,,w miejscu'', co redukuje złożoność pamięciową do $O(n)$.
\end{itemize}

\subsection{Zadanie 2: Wartość wielomianu}
\begin{itemize}
    \item \textbf{Funkcja:} \code{warNewton(x, fx, t)}
    \item \textbf{Opis:} Oblicza wartość wielomianu w punkcie $t$.
    \item \textbf{Optymalizacja:} Zastosowano uogólniony schemat Hornera, co pozwala na obliczenie wartości w czasie $O(n)$ bez jawnego potęgowania.
\end{itemize}

\subsection{Zadanie 3: Postać naturalna}
\begin{itemize}
    \item \textbf{Funkcja:} \code{naturalna(x, fx)}
    \item \textbf{Opis:} Przelicza współczynniki z bazy Newtona na bazę naturalną ($a_n x^n + \dots + a_0$).
    \item \textbf{Działanie:} Algorytm rekurencyjnie wymnaża czynniki liniowe $(x - x_k)$. Czas działania to $O(n^2)$.
\end{itemize}

\subsection{Zadanie 4: Rysowanie interpolacji}
\begin{itemize}
    \item \textbf{Funkcja:} \code{rysujNnfx(f, a, b, n; wezly=:rownoodlegle)}
    \item \textbf{Opis:} Rysuje wykres funkcji $f(x)$ i jej wielomianu interpolacyjnego Newtona (stopnia $n$) na przedziale $[a, b]$.
    \item \textbf{Węzły:} Możliwość wyboru węzłów równoodległych lub Czebyszewa. Funkcja wykorzystuje \code{ilorazyRoznicowe} i \code{warNewton}. Do wizualizacji użyto pakietu \code{Plots}.
\end{itemize}

\taskseparator

% -----------------------------------------------------------------------------
% WERYFIKACJA
% -----------------------------------------------------------------------------

\section{Weryfikacja poprawności}

Algorytmy przetestowano na wielomianie $P(x) = 2x^2 + 3x + 1$. Program poprawnie wyznaczył:
\begin{itemize}
    \item ilorazy różnicowe: \code{[0.0, 1.0, 2.0]},
    \item wartość w punkcie $x=2$: wynik \code{15.0},
    \item współczynniki naturalne: \code{[1.0, 3.0, 2.0]}.
\end{itemize}

\subsection{Wykresy testowe}

\begin{figure}[H]
    \centering
    \begin{subfigure}[b]{0.48\textwidth}
        \centering
        \includegraphics[width=\linewidth]{wykres_rownoodlegle.png}
        \caption{Węzły równoodległe (n=10)}
    \end{subfigure}
    \hfill
    \begin{subfigure}[b]{0.48\textwidth}
        \centering
        \includegraphics[width=\linewidth]{wykres_czebyszew.png}
        \caption{Węzły Czebyszewa (n=10)}
    \end{subfigure}
    \caption{Porównanie interpolacji dla wielomianu testowego}
\end{figure}

\subsection{Wnioski}
\begin{itemize}
    \item Algorytmy zostały poprawnie zaimplementowane.
    \item Na wykresach widać, że przy zastosowaniu węzłów równoodległych pojawiają się wyraźne błędy (oscylacje) na krańcach przedziału.
    \item Zastosowanie węzłów Czebyszewa znacząco poprawia wynik. Wynika to z faktu, że węzły te są zagęszczone na brzegach przedziału, co pozwala wielomianowi lepiej dopasować się do funkcji w tych newralgicznych miejscach i eliminuje duże odchylenia.
\end{itemize}

\taskseparator

% -----------------------------------------------------------------------------
% ZADANIE 5
% -----------------------------------------------------------------------------

\section{Zadanie 5: Zbieżność dla funkcji gładkich}

Zbadano zachowanie interpolacji na węzłach równoodległych dla funkcji klasy $C^\infty$ (gładkich). Sprawdzono, czy błąd maleje wraz ze wzrostem stopnia wielomianu ($n = 5, 10, 15$).

\subsection{Przypadek A: $f(x) = e^x$ na przedziale $[0, 1]$}

\begin{figure}[H]
    \centering
    \begin{subfigure}[b]{0.32\textwidth}
        \centering
        \includegraphics[width=\linewidth]{zad5_a_n5.png}
        \caption{$n=5$}
    \end{subfigure}
    \hfill
    \begin{subfigure}[b]{0.32\textwidth}
        \centering
        \includegraphics[width=\linewidth]{zad5_a_n10.png}
        \caption{$n=10$}
    \end{subfigure}
    \hfill
    \begin{subfigure}[b]{0.32\textwidth}
        \centering
        \includegraphics[width=\linewidth]{zad5_a_n15.png}
        \caption{$n=15$}
    \end{subfigure}
    \caption{Interpolacja $e^x$, węzły równoodległe}
\end{figure}

\noindent \textbf{Obserwacje:} \\
Funkcja $e^x$ jest bardzo regularna. Już dla $n=5$ przybliżenie jest dobre. Wraz ze wzrostem $n$ wielomian interpolacyjny idealnie pokrywa się z funkcją. Nie widać żadnych negatywnych efektów na brzegach.

\subsection{Przypadek B: $f(x) = x^2 \sin(x)$ na przedziale $[-1, 1]$}

\begin{figure}[H]
    \centering
    \begin{subfigure}[b]{0.32\textwidth}
        \centering
        \includegraphics[width=\linewidth]{zad5_b_n5.png}
        \caption{$n=5$}
    \end{subfigure}
    \hfill
    \begin{subfigure}[b]{0.32\textwidth}
        \centering
        \includegraphics[width=\linewidth]{zad5_b_n10.png}
        \caption{$n=10$}
    \end{subfigure}
    \hfill
    \begin{subfigure}[b]{0.32\textwidth}
        \centering
        \includegraphics[width=\linewidth]{zad5_b_n15.png}
        \caption{$n=15$}
    \end{subfigure}
    \caption{Interpolacja $x^2 \sin(x)$, węzły równoodległe}
\end{figure}

\noindent \textbf{Obserwacje:} \\
Mimo bardziej ,,falistego'' kształtu funkcji, interpolacja na węzłach równoodległych działa poprawnie. Zwiększanie liczby węzłów ($n$) systematycznie zmniejsza błąd interpolacji. Dla $n=15$ wykresy są niemal nierozróżnialne.

\subsection{Wniosek z Zadania 5}
Dla funkcji gładkich na krótkich przedziałach, interpolacja na węzłach równoodległych jest zbieżna i skuteczna.

\taskseparator

% -----------------------------------------------------------------------------
% ZADANIE 6
% -----------------------------------------------------------------------------

\section{Zadanie 6: Granice interpolacji (Zjawisko Rungego)}

Zbadano funkcje trudne, porównując węzły równoodległe (R) i Czebyszewa (C) dla stopni $n = 5, 10, 15$.

\subsection{Przypadek A: $f(x) = |x|$ na przedziale $[-1, 1]$ (Ostrze)}

\subsubsection*{Węzły Równoodległe}
\begin{figure}[H]
    \centering
    \begin{subfigure}[b]{0.32\textwidth}
        \centering
        \includegraphics[width=\linewidth]{zad6_a_n5_rown.png}
        \caption{$n=5$}
    \end{subfigure}
    \hfill
    \begin{subfigure}[b]{0.32\textwidth}
        \centering
        \includegraphics[width=\linewidth]{zad6_a_n10_rown.png}
        \caption{$n=10$}
    \end{subfigure}
    \hfill
    \begin{subfigure}[b]{0.32\textwidth}
        \centering
        \includegraphics[width=\linewidth]{zad6_a_n15_rown.png}
        \caption{$n=15$}
    \end{subfigure}
    \caption{Funkcja $|x|$ - Węzły równoodległe}
\end{figure}

\subsubsection*{Węzły Czebyszewa}
\begin{figure}[H]
    \centering
    \begin{subfigure}[b]{0.32\textwidth}
        \centering
        \includegraphics[width=\linewidth]{zad6_a_n5_czeb.png}
        \caption{$n=5$}
    \end{subfigure}
    \hfill
    \begin{subfigure}[b]{0.32\textwidth}
        \centering
        \includegraphics[width=\linewidth]{zad6_a_n10_czeb.png}
        \caption{$n=10$}
    \end{subfigure}
    \hfill
    \begin{subfigure}[b]{0.32\textwidth}
        \centering
        \includegraphics[width=\linewidth]{zad6_a_n15_czeb.png}
        \caption{$n=15$}
    \end{subfigure}
    \caption{Funkcja $|x|$ - Węzły Czebyszewa}
\end{figure}

\noindent \textbf{Obserwacje:}
\begin{itemize}
    \item Dla węzłów równoodległych, próba odwzorowania szpica w $x=0$ powoduje zafalowanie wielomianu na bokach.
    \item Dla węzłów Czebyszewa, wielomian znacznie lepiej radzi sobie z ostrzem, a błąd na reszcie przedziału jest minimalny.
\end{itemize}

\subsection{Przypadek B: $f(x) = \frac{1}{1 + x^2}$ na przedziale $[-5, 5]$ (Funkcja Rungego)}

To jest kluczowy test stabilności interpolacji.

\subsubsection*{Ewolucja błędu dla węzłów Równoodległych}
\begin{figure}[H]
    \centering
    \begin{subfigure}[b]{0.32\textwidth}
        \centering
        \includegraphics[width=\linewidth]{zad6_b_n5_rown.png}
        \caption{$n=5$}
    \end{subfigure}
    \hfill
    \begin{subfigure}[b]{0.32\textwidth}
        \centering
        \includegraphics[width=\linewidth]{zad6_b_n10_rown.png}
        \caption{$n=10$}
    \end{subfigure}
    \hfill
    \begin{subfigure}[b]{0.32\textwidth}
        \centering
        \includegraphics[width=\linewidth]{zad6_b_n15_rown.png}
        \caption{$n=15$}
    \end{subfigure}
    \caption{Runge - Węzły równoodległe}
\end{figure}

\subsubsection*{Rozwiązanie problemu - Węzły Czebyszewa}
\begin{figure}[H]
    \centering
    \begin{subfigure}[b]{0.32\textwidth}
        \centering
        \includegraphics[width=\linewidth]{zad6_b_n5_czeb.png}
        \caption{$n=5$}
    \end{subfigure}
    \hfill
    \begin{subfigure}[b]{0.32\textwidth}
        \centering
        \includegraphics[width=\linewidth]{zad6_b_n10_czeb.png}
        \caption{$n=10$}
    \end{subfigure}
    \hfill
    \begin{subfigure}[b]{0.32\textwidth}
        \centering
        \includegraphics[width=\linewidth]{zad6_b_n15_czeb.png}
        \caption{$n=15$}
    \end{subfigure}
    \caption{Runge - Węzły Czebyszewa}
\end{figure}

\subsection{Analiza}
To klasyczny przykład testujący stabilność interpolacji.
\begin{itemize}
    \item \textbf{Węzły równoodległe:} Obserwujemy klasyczne \textbf{Zjawisko Rungego}. Dla $n=10$ i $n=15$ wielomian wpada w silne oscylacje na krańcach przedziału, osiągając wartości wielokrotnie przekraczające zakres funkcji. Interpolacja jest rozbieżna, co wynika z niekorzystnego układu węzłów powodującego gwałtowny wzrost czynnika iloczynowego $\prod(x-x_i)$ we wzorze na błąd.
    \item \textbf{Węzły Czebyszewa:} Interpolacja jest zbieżna. Zastosowanie węzłów będących zerami wielomianu Czebyszewa zagęszcza punkty pomiarowe na krańcach przedziału. Zgodnie z teorią, minimalizuje to normę czynnika wielomianowego w oszacowaniu błędu, co skutecznie eliminuje oscylacje Rungego. Wielomian dla $n=15$ niemal idealnie pokrywa się z zadaną funkcją.
\end{itemize}

\subsection{Wnioski końcowe}
Wybór rodzaju węzłów ma kluczowe znaczenie dla zbieżności interpolacji wielomianowej.
\begin{enumerate}
    \item Dla funkcji analitycznych (jak w przypadku Rungego), węzły równoodległe mogą prowadzić do rozbieżności (dużych oscylacji na brzegach) wraz ze wzrostem stopnia wielomianu.
    \item Węzły Czebyszewa są rozwiązaniem optymalnym – poprzez zagęszczenie na krańcach minimalizują błąd maksymalny i gwarantują zbieżność jednostajną, eliminując zjawisko Rungego.
\end{enumerate}

\end{document}